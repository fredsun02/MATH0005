\documentclass[11pt]{article}
\usepackage[utf8]{inputenc}	% Para caracteres en español
\usepackage{amsmath,amsthm,amsfonts,amssymb,amscd}
\usepackage{multirow,booktabs}
\usepackage[table]{xcolor}
\usepackage{fullpage}
\usepackage{lastpage}
\usepackage{enumitem}
\usepackage{fancyhdr}
\usepackage{mathrsfs}
\usepackage{wrapfig}
\usepackage{setspace}
\usepackage{calc}
\usepackage{verbatim}
\usepackage{float}
\usepackage{multicol}
\usepackage{cancel}
\usepackage[retainorgcmds]{IEEEtrantools}
\usepackage[margin=3cm]{geometry}
\usepackage{amsmath}
\newlength{\tabcont}
\setlength{\parindent}{0.0in}
\setlength{\parskip}{0.05in}
\usepackage{empheq}
\usepackage{framed}
\usepackage[most]{tcolorbox}
\usepackage{xcolor}
\colorlet{shadecolor}{orange!15}
\parindent 0in
\parskip 12pt
\geometry{margin=1in, headsep=0.25in}
\theoremstyle{definition}
\newtheorem{defn}{Definition}[subsection]
\newtheorem{reg}{Rule}
\newtheorem{exer}{Exercise}[subsection]
\newtheorem{note}{Note}[subsection]
\newtheorem{theorem}{Theorem}[subsection]
\newtheorem{eg}{Example}[subsection]
\begin{document}
\setcounter{section}{0}
\title{MATH0005 NOTES}

\thispagestyle{empty}

\begin{center}
{\LARGE \bf MATH0005 Algebra 1}\\
%{\large \bf 1 Logic}\\
\quad\\
{\Large \bf Fall 2022}
\end{center}
\tableofcontents
\newpage

\section{Logic}
\subsection{Propositional calculus}
\begin{shaded}
\begin{defn}
    A \textbf{proposition} is a mathematical statement which is either true or false.
\end{defn}
\end{shaded}
Here are some example propositions:
    \begin{itemize}
        \item 123456 is a prime number.
        \item The function $f(x)= \sin x$ is continuous.
        \item $1+1=2$
        \item The Riemann hypothesis is false.
    \end{itemize}
Proposition might be true or false. All the propositions have their \textit{well-defined truth value}, even if we do not know what it is. One typical example showing a sentence is not a proposition is, ``$n$ is even'', since it does not have a \textit{well-defined truth value} until we know what $n$ is.

\begin{shaded}
\begin{defn}
\textbf{Truth value} is the attribute of a proposition as to whether the proposition is true or false.
\end{defn}
\end{shaded}

Here is an example:
\begin{itemize}
    \item The \textit{truth value} for ``$7$ is odd'' is \textit{true}, which can be denoted as \textbf{T}.
    \item The \textit{truth value} for ``$7$ is even'' is \textit{false}, which can be denoted as \textbf{F}.
\end{itemize}

\subsection{Logical connectives}
\begin{shaded}
\begin{defn}
    \textbf{Connectives} are symbols used to connect two or more propositional or predicate logics in such a manner that resultant logic depends only on the input logics and the meaning of the connective used.
\end{defn}    
\end{shaded}

The standard connectives introduced here are `\textit{and}', `\textit{or}', `\textit{not}', `\textit{implies}' and `\textit{if and only if}' (iff, for short).

Here are some examples of propositions which contain connectives:
\begin{itemize}
    \item \textbf{AND} ``34043 is a sum of two squares \textit{and} 34043 is divisible by 17''
    \item \textbf{OR} ``34043 is a sum of two squares \textit{or} 34043 is divisible by 17''
    \item \textbf{NOT} ``It is \textit{not} true that 34043 is a sum of two squares''
    \item \textbf{IMPLIES} ``34043 is odd \textit{implies} 34043 is divisible by 3''
    \item \textbf{IF AND ONLY IF} ``An odd number is a sum of two squares, \textit{if and only if} it leaves remainder 1 when you divide it by 4''
\end{itemize}

Notice that \textbf{implies} is often express as ``\textbf{if $\dots$ then}''.

The sentence ``34043 is odd \textit{implies} 34043 is divisible by 3'' could be written as ``\textit{If} 34043 is odd \textit{then} 34043 is divisible by 3''.

\subsection{Well-formed formulae}
\subsubsection{Variables and connective symbols}
\begin{shaded}
\begin{defn}
    \textbf{Propositional variables} are symbols that represent a proposition.
\end{defn}    
\end{shaded}
Traditionally, we use lower case English letters like $p,q,r,\dots$ or letters with subscripts $p_{1},p_{2}$ to represent propositional variables.

For the logical connectives we discussed before, we also have symbols.
\begin{shaded}
\textbf{Connectives symbols}
\begin{itemize}
    \item $\wedge$ represents \textit{and}.
    \item $\vee$ represents \textit{or}.
    \item $\Rightarrow or \rightarrow$ represents \textit{implies}.
    \item $\neg$ represents \textit{not}.
    \item Besides, we also use brackets: $($ and $)$.
\end{itemize}    
\end{shaded}
By collections of propositional variables, connectives and brackets, we could start to build formulae.
\begin{note}
    \textbf{Well-formed formulae} are defined as the collections of those logical propositions with sensible meaning.
\end{note}

\subsubsection{Definition of a well-formed formula}
By definition, something is a \textbf{Well-formed formula}, or \textbf{WFF} for short, if and only if it can be constructed using these rules below:
\begin{shaded}
\textbf{Rules of WFFs}
\begin{enumerate}
    \item A propositional variable is a WFF.
    \item If $\phi$ and $\psi$ are any two WFFs, then
        \begin{enumerate}
            \item $(\phi \wedge \psi)$ is a WFF,
            \item $(\phi \vee \psi)$ is a WFF,
            \item $(\phi \Rightarrow \psi)$ is a WFF
            \item $\neg \phi$ is a WFF.
        \end{enumerate}
\end{enumerate}
\end{shaded}

\subsubsection{WFF examples}
Here are some examples of WFFs.\\
Suppose we have $p$ and $q$ as propositional variables. Then we have following WFFs:
\begin{itemize}
    \item $p$ is a WFF (from \textit{Rule 1}).
    \item $(p \Rightarrow q)$ is a WFF (from \textit{Rule 1 twice then Rule 2.c}).
    \item $\neg q$ is a WFF (from \textit{Rule 1 then Rule 2.d}).
    \item $((p \Rightarrow q) \vee \neg q)$ is a WFF. Since $(p \Rightarrow q)$ and $\neg q$ are WFFs, and by \textit{Rule 2.b} the whole thing is a WFF.
    \item $\neg \neg (p \Rightarrow q)$ is a WFF. Since $(p \Rightarrow q)$ is a WFF, by using \textit{Rule 2.d} twice, the whole thing is a WFF.
\end{itemize}

Notice that things can be WFFs \textbf{only} when using those rules. Thereby formulae below are not WFFs:
\begin{itemize}
    \item $p \vee q$, missing the brackets $()$.
    \item $(p \wedge q \wedge r)$. It should be corrected as $((p \wedge q) \wedge r)$ or $(p \wedge (q \wedge r))$.
\end{itemize}

\subsection{Truth tables}
Notice that a \textbf{Well-formed formula} like $(p \wedge q)$ is not true or false on its own: it totally depends on the truth or falsity of the statements represented by the propositional variables $p$ and $q$. Once we decide the truth value of propositional variables in a WFF, we can give a truth value to the whole WFF.

\begin{note}
    A \textbf{Truth-table} shows how the truth value of a WFF depends on its propositional variables and connectives.
\end{note}

\subsubsection{Truth assignments for propositional variables}
\begin{shaded}
\begin{defn}
    A \textbf{truth assignment} for a set $V$ of propositional variables is a function $v: V \rightarrow \left\{T,F\right\}$.
\end{defn}
\end{shaded}
Here and elsewhere, $T$ represents true and $F$ represents false.
\begin{eg}
    If $p$ and $q$ are propositional variables and $V = \left\{p,q\right\}$, then there is a truth assignment $v$ for $V$ such that $v(p)=T$ and $v(q)=F$.
\end{eg}

\subsubsection{Extending a truth assignment to WFFs}
The truth assignment to WFF is more complex compared with assignments to propositional variables themselves, since it also depends on the intended meaning of the logical connectives.

Consider the following WFF:
\begin{equation}
    ((p \Rightarrow (q \vee r)) \Rightarrow (\neg p \vee q))
\end{equation}

Suppose that we somehow already knew the truth values of $p,q,r$, we still need to know about the truth value of formulae connected with those connectives $\wedge, \vee, \Rightarrow and \neg$.

Let us start with the connective $\wedge$ (\textit{and}):

Since $\wedge$ represents the ordinary usage of the word ``and'', it would be sensible to assign $(\psi \wedge \psi)$ is true if and only if both $\phi$ and $\psi$ are true.

Here we have the \textbf{truth table} for $\wedge$:
\begin{table}[H]
    \centering
    \begin{tabular}{|c|c|c|}
    \hline
    \textbf{$\phi$} & \textbf{$\psi$} & \textbf{$(\phi \wedge \psi)$} \\ \hline
    T          & T          & T          \\ \hline
    T          & F          & F          \\ \hline
    F          & T          & F          \\ \hline
    F          & F          & F          \\ \hline
    \end{tabular}
    \caption[]{Truth table for $\wedge$}
\end{table}

By using the truth table above, we could assign the truth value to a WFF $(\phi \wedge \psi)$.

Here are some examples:

\begin{eg}
    If $v(\psi)=T$ and $v(\phi)=T$, then $v(\psi \wedge \phi)$ will be $T$.
\end{eg}

\begin{eg}
    If the propositional variables ``$p:$ \textit{Today is Thursday}'' and ``$q:$ \textit{Today is raining}'' are true, then the proposition ``\textit{Today is Thursday and is raining}'' is true.
\end{eg}

Here are the truth tables for other connectives in our language.
\begin{table}[H]
    \centering
    \begin{tabular}{|c|c|c|}
    \hline
    \textbf{$\phi$} & \textbf{$\psi$} & \textbf{$(\phi \vee \psi)$} \\ \hline
    T          & T          & T          \\ \hline
    T          & F          & T          \\ \hline
    F          & T          & T          \\ \hline
    F          & F          & F          \\ \hline
    \end{tabular}
    \caption[]{Truth table for $\vee$}
\end{table}

\begin{table}[H]
    \centering
    \begin{tabular}{|c|c|c|}
    \hline
    \textbf{$\phi$} & \textbf{$\psi$} & \textbf{$(\phi \Rightarrow \psi)$} \\ \hline
    T          & T          & T          \\ \hline
    T          & F          & F          \\ \hline
    F          & T          & T          \\ \hline
    F          & F          & T          \\ \hline
    \end{tabular}
    \caption[]{Truth table for $\Rightarrow$}
\end{table}

\begin{table}[H]
    \centering
    \begin{tabular}{|c|c|}
    \hline
    \textbf{$\phi$} & \textbf{$(\neg \phi)$} \\ \hline
    T          & F          \\ \hline
    F          & T          \\ \hline
    \end{tabular}
    \caption[]{Truth table for $\neg$}
\end{table}

Notices that for all truth assignments we could think about as a way to combine two truth values into another truth value, just like $+$ combines two numbers into another number. Thereby we could have this conclusion:
\begin{itemize}
    \item $T \wedge T = T$, $T \wedge F = F$, $F \wedge T = F$, $F \wedge F = F$
    \item $T \vee T = T$, $T \vee F = T$, $F \vee T = T$, $F \vee F = F$
    \item $T \Rightarrow T = T$, $T \Rightarrow F = F$, $F \Rightarrow T = T$, $F \Rightarrow F = T$
    \item $\neg T = F$, $\neg F = T$
\end{itemize}

\subsection{Truth values for WFFs}
For a truth assignment $v: V \rightarrow \left\{T,F\right\}$, we could extend the function $v$ so that it gives a truth value to any WFF using the propositional variables V such that for any WFFs $\phi$ and $\psi$.
\begin{gather}
    v((\phi \wedge \psi)) = v(\phi) \wedge v(\phi)\\
    v((\phi \vee \psi)) = v(\phi) \vee v(\psi)\\
    v((\phi \Rightarrow \psi)) = v(\phi) \Rightarrow v(\psi)\\
    v((\neg \phi)) = \neg v(\phi)
\end{gather}

This could support the idea that connective symbols are not just as parts of WFFs but as ways of combing truth values. Here are some examples:

Suppose $V \rightarrow \left\{\phi,\psi\right\}$ and $v(\phi)=T$, $v(\psi)=F$, we could have
\begin{eg}
    \begin{equation}
        \begin{split}
            v((\phi \wedge \psi)) &= v(\phi) \wedge v(\phi)\\
                                  &= T \wedge F\\
                                  &= F
        \end{split}
    \end{equation}
\end{eg}

\begin{eg}
    \begin{equation}
        \begin{split}
             v((\neg \phi) \Rightarrow (\phi \vee \psi)) &= v(\neg \phi) \Rightarrow v(\phi \vee \psi)\\
                                                           &= \neg v(\phi) \Rightarrow (v(\phi) \vee v(\psi))\\
                                                           &= F \Rightarrow (T \vee F)\\
                                                           &= F \Rightarrow T\\
                                                           &= T
        \end{split}
    \end{equation}    
\end{eg}

\begin{eg}
    Let $\alpha = ((p \wedge q) \vee (\neg p \wedge \neg q))$, $v(p)=T$, $v(q)=F$. Let us find $v(\alpha)$.\\
    Firstly let expand $v(\alpha)$ as $v(((p \wedge q) \vee (\neg p \wedge \neg q))$,\\
    We have:
    \begin{equation}
        \begin{split}
            v(\alpha)&= v(((p \wedge q) \vee (\neg p \wedge \neg q)) \\
                     &= v(p \wedge q) \vee v(\neg p \wedge \neg q)\\
                     &= (v(p) \wedge v(q)) \vee (v(\neg p) \wedge v(\neg q))\\
                     &= (v(p) \wedge v(q)) \vee (\neg v(p) \wedge \neg v(q))\\
                     &= (T \wedge F) \vee (\neg T \wedge \neg F)\\
                     &= F \vee (F \wedge T)\\
                     &= F \vee F\\
                     &= F
        \end{split}
    \end{equation}
\end{eg}

\begin{eg}
    Consider a WFF $\phi=(p \Rightarrow (p \Rightarrow p))$ and $v(p)=T$. Let find $v(\phi)$.
    We have:
    \begin{equation}
        \begin{split}
            v(\phi) &= v(p \Rightarrow (p \Rightarrow p))\\
                    &= v(p) \Rightarrow (v(p \Rightarrow p))\\
                    &= v(p) \Rightarrow (v(p) \Rightarrow v(p))\\
                    &= T \Rightarrow (T \Rightarrow T)\\
                    &= T \Rightarrow T\\
                    &= T
        \end{split}
    \end{equation}
    If we work out the truth value of $\phi$ when $v(p)=F$, we would find the result of $v(\phi)$ is also $T$:
    \begin{equation}
        \begin{split}
            v(\phi) &= v(p \Rightarrow (p \Rightarrow p))\\
                    &= v(p) \Rightarrow (v(p \Rightarrow p))\\
                    &= v(p) \Rightarrow (v(p) \Rightarrow v(p))\\
                    &= F \Rightarrow (F \Rightarrow F)\\
                    &= F \Rightarrow T\\
                    &= T
        \end{split}
    \end{equation}
\end{eg}

We would observe the WFF $\phi$ is true for \textit{every} truth assignment of its variables, and we call WFF with this property as \textbf{tautology}.

\begin{shaded}
\begin{defn}
    A formula of propositional logic is a \textbf{tautology} if the formula itself is always \textit{true}, regardless of which valuation is used for the propositional variables. 
\end{defn}
\end{shaded}

Here are some examples of tautology:
\begin{itemize}
    \item $(A \vee \neg A)$
    \item $((A \Rightarrow B) \wedge (B \Rightarrow C)) \Rightarrow (A \Rightarrow C)$
\end{itemize}

Similarly, we would have another kind of WFFs for which is false for \textit{every} truth assignment of its variables, and we call WFF with this property as \textbf{contradiction}.

\begin{shaded}
\begin{defn}
    A formula of propositional logic is a \textbf{contradiction} if the formula itself is always \textit{false}, regardless of which valuation is used for the propositional variables. 
\end{defn}
\end{shaded}

Here are some examples of tautology:
\begin{itemize}
    \item $(A \wedge \neg A)$
    \item $\neg (p \Rightarrow (q \Rightarrow p))$
\end{itemize}

\subsection{Logical equivalence}
Consider the following two WFFs:
\begin{gather}
    \phi = (p \wedge q)\\
    \psi = (q \wedge p)
\end{gather}

The formulae $\phi$ and $\psi$ are \textit{different} WFFs. However, no matter what truth assignment we give, the formulae $\phi$ and $\psi$ could always get equal truth values. Here is the proof:

\begin{equation}
    \begin{split}
        v(\phi) &= v(p \wedge q)\\
                &= v(p) \wedge v(q)
    \end{split}
\end{equation}

\begin{equation}
    \begin{split}
        v(\psi) &= v(q \wedge p)\\
                &= v(q) \wedge v(p)
    \end{split}
\end{equation}

And no matter what $v(q)$ and $v(p)$ is, $v(p) \wedge v(q)$ always equals to $v(q) \wedge v(p)$.

\begin{shaded}
\begin{defn}
    Two WFFs $\phi$ and $\psi$ are called \textbf{logically equivalent}, denoted as $\phi \equiv \psi$, if and only if they have the same truth value under every possible truth assignment. 
\end{defn}
\end{shaded}

The idea of logically equivalence is useful for us to prove something is true, because we can prove some logically equivalent formula instead if it is easier to interpret.
\begin{shaded}
\begin{theorem}
    \textit{Let $\phi$, $\psi$ and $\theta$ be WFFs, then}
    \begin{enumerate}
        \item $(\phi \wedge \psi) \equiv (\psi \wedge \phi)$ \textit{[Commutativity properties for $\wedge$]}
        \item $(\phi \vee \psi) \equiv (\psi \vee \phi)$ \textit{[Commutativity properties for $\vee$]}
        \item $(\phi \wedge (\psi \wedge \theta) \equiv ((\psi \wedge \phi) \wedge \theta)$ \textit{[Associativity Properties]}
        \item $(\phi \vee (\psi \vee \theta) \equiv ((\psi \vee \phi) \vee \theta)$ \textit{[Associativity Properties]}
        \item $(\phi \wedge (\psi \vee \theta)) \equiv ((\phi \wedge \psi) \vee (\phi \wedge \theta)))$ \textit{[Distributivity]}
        \item $(\phi \vee (\psi \wedge \theta)) \equiv ((\phi \vee \psi) \wedge (\phi \vee \theta)))$ \textit{[Distributivity]}
    \end{enumerate}
\end{theorem}
\end{shaded}

\subsection{Double negation}
\begin{shaded}
\begin{theorem}
    \textit{Let $\phi$ be WFF, then}
    \begin{equation}
        \neg \neg \phi \equiv \phi
    \end{equation}
\end{theorem}
\end{shaded}
\begin{proof}
    Define function $v: V \rightarrow \left\{T,F\right\}$.\\
    If $v(\phi)=T$, then $v(\neg \neg \phi)=\neg (\neg v(\phi))=\neg (\neg T)=\neg F=T$;\\
    If $v(\phi)=F$, then $v(\neg \neg \phi)=\neg (\neg v(\phi))=\neg (\neg F)=\neg T=F$.\\
    Therefore under any truth assignment $v$ we have $v(\phi)=v(\neg \neg \phi)$.
\end{proof}

\subsection{De Morgan's Laws}
\begin{shaded}
\begin{theorem}
    \textit{Let $\phi$ and $\psi$ be WFFs, then}
    \begin{enumerate}
        \item $\neg (\phi \wedge \psi)=(\neg \phi \vee \neg \psi)$
        \item $\neg (\phi \vee \psi)=(\neg \phi \wedge \neg \psi)$
    \end{enumerate}
\end{theorem}
\end{shaded}

\begin{proof}
    By checking the possibilities for the truth values of $\phi$ and $\psi$ under any assignment.
\end{proof}

Notice that \textbf{De Morgan's Laws} can be generalised to more than two WFFs.
\begin{shaded}
\begin{theorem}
    \textit{Let $\phi$ and $\psi$ be WFFs, then}
    \begin{enumerate}
        \item $\neg (\phi_{1} \wedge \dots \wedge \phi_{n}) \equiv (\neg \phi_{1} \vee \dots \vee \neg \phi_{n})$
        \item $\neg (\phi_{1} \vee \dots \vee \phi_{n}) \equiv (\neg \phi_{1} \wedge \dots \neg \phi_{n})$
    \end{enumerate}
\end{theorem}
\end{shaded}

\subsection{Contrapositive}
\subsubsection{Equivalence between $\Rightarrow$ and combination of $\neg$ and $\vee$}
\begin{shaded}
    \begin{theorem}
        \textit{Let $\phi$ and $\psi$ be WFFs, then}
        \begin{equation}
            (\phi \Rightarrow \psi) \equiv (\neg \phi \vee \psi) \label{Rightarrowreplacement}
        \end{equation}
    \end{theorem}
\end{shaded}
\begin{proof}
    \quad \\
    Suppose that if $\phi$ is true, then $v(\phi \Rightarrow \psi)=T$ when $v(\psi)=T$, and $v(\phi \Rightarrow \psi)=F$ when $v(\psi)=F$. Similarly when $\phi$ is false,  then $v(\phi \Rightarrow \psi) \equiv T$ regardless of the truth value of $\psi$.\\
    On the other hand if $\phi$ is true, then $\neg \phi$ is false. When $\psi$ is true $v(\neg \phi \vee \psi)=T$. When $\psi$ is false  $v(\neg \phi \vee \psi)=F$. Similarly when $\phi$ is false, $\neg \phi$ is true. Therefore no matter what the truth value of $\psi$ is, $v(\neg \phi \vee \psi) \equiv T$.\\
    Therefore, 
    \begin{equation}
        (\phi \Rightarrow \psi) \equiv (\neg \phi \vee \psi)
    \end{equation}
\end{proof}

\subsubsection{Contrapositive}
\begin{shaded}
    \begin{defn}
        \textbf{Contrapositive} of an implication $A \Rightarrow B$ is defined as $\neg B \Rightarrow \neg A$. 
    \end{defn}
\end{shaded}
Regardless of the truth value of $A$ and $B$, the truth value of a proposition always equals to the truth value of its contrapositive proposition.\\
Here is a proof.
\begin{proof}
    \begin{equation}
        \begin{split}
            (A \Rightarrow B) & \equiv (A \vee \neg B)\\
                              & \equiv ((\neg \neg A) \vee \neg B)\\
                              & \equiv (\neg A \Rightarrow \neg B)
        \end{split}
    \end{equation}
\end{proof}
The equivalence between a proposition and its contrapositive proposition provides a useful tool to simplify some proof processes. It is sometimes easier to proof the contrapositive $\neg B \Rightarrow \neg A$ compared with the original one. We here have one example to support:
\begin{center}
    $x^2$ is an irrational number implies $x$ is an irrational number.\\
    (\textit{$\forall x^2 \in (\mathbb{R} - \mathbb{Q}) \Rightarrow x \in (\mathbb{R} - \mathbb{Q})$ })
\end{center}

The contrapositive of this proposition is:
\begin{center}
    $x$ is a rational number implies $x^2$ is a rational number.\\
    (\textit{$\forall x \in \mathbb{R} \Rightarrow x^2 \in \mathbb{R}$})
\end{center}

It is clear that the contrapositive is easier to prove because when $x$ is a rational number we know $x=p/q$. Therefore $x^2=p^2/q^2$ must be a rational number either.

\subsubsection{Converse}
\begin{shaded}
    \begin{defn}
        \textbf{Converse} of an implication $A \Rightarrow B$ is defined as $ B \Rightarrow A$. 
    \end{defn}
\end{shaded}
Unlike contrapositive, the proposition and its converse is \textbf{not} in general logically equivalent. Here is a contradiction example:
\begin{proof}
    \textit{Suppose ($A \Rightarrow B$) $\equiv$ ($B \Rightarrow A$):}
    When $v(A)=T$ and $v(B)=F$,
    \begin{equation}
        v(A \Rightarrow B) = T \Rightarrow F = F.
    \end{equation}
    But
    \begin{equation}
        v(B \Rightarrow A) = F \Rightarrow T = T.
    \end{equation}
    Therefore,
    ($A \Rightarrow B$) $\not \equiv$ ($B \Rightarrow A$)
\end{proof}

\subsection{Adequacy}
\begin{shaded}
    \begin{defn}
        A set of connectives is \textbf{adequacy} if every WFF is logically equivalent to one using only the connectives from the set.
    \end{defn}
\end{shaded}
\textbf{Theorem 1.9.1. $            (\phi \Rightarrow \psi) \equiv (\neg \phi \vee \psi)$} shows how we use $\neg$ and $\vee$ to replace the symbol $\Rightarrow$.

By using this theorem, all the occurrence of $p \Rightarrow q$ could be replaced by $(\neg p) \vee q$. This shows that the set $\left\{ \wedge,\vee,\neg \right\}$ is adequate.

However, we could still find other even smaller adequate sets.

\begin{shaded}
    \begin{theorem}
        $\left\{ \vee, \neg \right\}$ is adequate.
    \end{theorem}
\end{shaded}

\begin{proof}
    \quad \\
    Since we have already know the set $\left\{ \wedge,\vee,\neg \right\}$ is adequate, according to the \textbf{De Morgan's Laws}, all the occurrence of $(p \wedge q)$ could be replaced by $\neg ((\neg p) \vee (\neg q))$, which only contains the symbols $\neg$ and $\vee$.\\
    Therefore, $\left\{ \vee, \neg \right\}$ is adequate.
\end{proof}

Here is an example:
\begin{eg}
    Consider a proposition:
    \begin{equation}
        p \Rightarrow (q \wedge r)
    \end{equation}
    It could be replaced by the symbol $\vee$ and $\neg$.
    \begin{equation}
        \begin{split}
            p \Rightarrow (q \wedge r) & \equiv (\neg p) \vee (q \wedge r)\\
                                       & \equiv (\neg p) \vee \neg ((\neg q) \vee (\neg r)).
        \end{split}
    \end{equation}
\end{eg}

\begin{shaded}
    \begin{theorem}
        $\left\{ \wedge, \neg \right\}$ is adequate.
    \end{theorem}
\end{shaded}

\begin{proof}
    \quad \\
    The proof of this theorem is similar with the one above.\\
    According to \textbf{De Morgan's Laws}, all the occurrence of $(p \vee q)$ could be replaced by $\neg ((\neg p) \wedge (\neg q))$.
\end{proof}

\subsubsection{Do we have sets of connectives that are not adequate?}
It is clear the symbol $\wedge$, $\vee$ and $\Rightarrow$ is not adequate by themselves. But if we define new connectives like $p \uparrow q = \neg (p \wedge q)$ or $p \downarrow q = \neg (p \vee q)$, it can be shown that both $\left\{ \uparrow \right\}$ and $\left\{ \downarrow \right\}$ are adequate.
\newpage

\begin{comment}
\begin{note}
\textbf{Capital Letters refer to the accelerating reference frame \textit{S} while lowercase letters refer to the inertial reference frame S$_0$}
\end{note}
Picture a moving reference frame, \textit{S}, moving relative to S$_0$. Imagine in the the moving reference frame that a ball with mass, \textit{m} is being thrown. 
In order to consider the motion of the ball, the motion must be first considered in the inertial reference frame. 
\begin{equation}
F = m\ddot{r_0}
\end{equation}
Where r$_0$ is the ball's position relative to S$_0$. 

Now, by considering the motion of the ball in the accelerating frame, the ball position relative to \textit{S} is \textit{R}. (It's velocity is $\dot{R}$. 
Thus, relating \textit{R} to $r_0$, we have: 
\begin{equation}
\dot{r_0} = \dot{R} + V
\end{equation}
Newton's second law for the inertial reference frame by differentiate and multiplying by mass is:
\begin{equation}
F_{\text{inertial}} = -mA = -m\ddot{R}
\end{equation}
\subsection{The Tides}
\begin{shaded}
\textbf{The Tidal Force} \newline
\begin{equation}
F_{tide} = -GM_mm(\frac{\hat{d}}{d^2}-\frac{\hat{d_0}}{d_0^2})
\end{equation}
Where:
\begin{equation*}
\begin{split}
G = \text{Gravitational Constant} \\
d = \text{Object's Position Relative to Moon} \\
d_0 = \text{Earth's Center Relative to the moon}\\
M_m = \text{Mass of the moon}
\end{split}
\end{equation*}
\end{shaded}
\newpage
\subsection{The Angular Velocity Vector}
The rest of the notes and the chapter will over reference frames that are rotating with respect to the inertial reference frame, so angular velocity has to be used. 
\begin{defn}
\textbf{Euler's Theorem} - The most general motion of any body relative to a fixed point \textit{O} is a rotation about some axis through \textit{O} To specify this rotation about a given point O, we only have to give the direction of the axis and the rate of rotation, or angular velocity $\omega$. Because this has a magnitude and direction, it is an obvious choice to write this rotation vector as $\omega$, the angular velocity vector. That is:
\begin{equation}
\omega = \omega\textbf{u}
\end{equation}
Where \textbf{u} is the unit vector
\end{defn}
\begin{shaded}
\textbf{Vector Velocity}\newline
The velocity at any point, \textit{P} (position, \textit{r}) is given by:
\begin{equation}
v = \omega\  x \ r
\end{equation}
\end{shaded}
\subsection*{Addition of Angular Velocities}
One can add angular velocities just like linear velocities. If body 3 is rotating at angular velocity $\omega_{32}$ relative to frame 2, and frame 2 is rotating at angular velocity $\omega_{21}$ relative to frame 1, then body 3 is rotating relative to frame 1 at angular velocity: 
\begin{equation}
\omega_{31} = \omega_{32} + \omega_{21}
\end{equation}
\subsection{Time Derivatives in Rotating Frames}
If frame S has a angular velocity, $\Omega$ relative to S$_0$ then the time derivative of a single vector \textbf{Q} as seen in the two frames are related by:
\begin{equation}
(\frac{d\textbf{Q}}{dt})_{S_0} = (\frac{d\textbf{Q}}{dt})_{S} \ + \Omega \ x \ \textbf{Q}
\end{equation}
\subsection{Netwon's Second Law in a Rotating Frame}
A particle in an inertial reference frame, S$_0$ obeys Newton's second law as we are use to:
\begin{equation}
m\frac{d^2r}{dt^2} = F
\end{equation}
Using the results from equation 8, the time derivative for a rotating frame with reference to an inertial frame can be given by:
\begin{equation}
(\frac{dr}{dt})_{S_0} = (\frac{dr}{dt})_s \ + \Omega \ x \ r
\end{equation}
By differentiation, Newton's second law becomes:
\begin{equation}
m\ddot{r} = F + 2m\dot{r} \ x \ \Omega \ + m(\Omega \ x \ r) \ x \ \Omega
\end{equation}
Where \textit{F} is the sum of all forces in the inertial reference frame. 
\subsection{The Centrifugal Force}
This is an inertial force in a rotating reference frame 
\begin{equation}
F_{\text{cf}} = m(\Omega \ x \ r) \ x \ \Omega
\end{equation}
\subsubsection*{Free-Fall Acceleration (Non-Vertical Gravity)}
\begin{equation}
F_{\text{eff}} = F_{\text{grav}} + F_{\text{cf}} = mg_0 + m\Omega^2R\sin(\theta)\hat{\rho}
\end{equation}
The acceleration due to the Centrifugal force is simply 
\begin{equation}
\begin{split}
g = g_0 + \Omega^2R\sin(\theta)\hat{\rho} \\
g_{\text{rad}} = g_0 - \Omega^2R\sin^2(\theta)  \\
g_{\text{tan}} = \Omega^2R\sin(\theta)\cos(\theta)
\end{split}
\end{equation}
The angle between g and its radial direction is:
\begin{equation}
\alpha \approx \frac{g_{\text{tan}}}{g_{\text{rad}}} 
\end{equation}
The maxium value at ($\theta$ = 45):
\begin{equation}
\alpha_{\text{max}} =  \frac{\Omega^2R}{2g_0}
\end{equation}
\subsection{Coriolis Force}
The Coriolis Force is another inertial force in a rotating reference frame that an object experiences when it is moving. 
\begin{equation}
F_{\text{cor}} = 2m\dot{r} \ x \ \Omega = 2mv \ x \ \Omega
\end{equation}
The maximum acceleration, \textit{a} that the Coriolis force could produce acting by itself with \textit{v} perpendicular to $\Omega$ is:
\begin{equation}
a_{\text{max}} = 2v\Omega 
\end{equation}
\begin{shaded}
\textbf{Direction of the Coriolis Force} \newline
The Direction of the Coriolis force us always perpendicular to the velocity of the object (hence equation 17), and is given by the right hand rule. 
\end{shaded}
\newpage
\subsection{Free Fall and the Coriolis Force}
\begin{equation}
m\ddot{r} = mg_0 + F_{\text{cf}} + F_{\text{cor}} 
\end{equation}
\subsection{The Foucault Pendulum}
See chapter 9, Page 354. There is no need to recopy what is in the book here. 
\end{comment}
\end{document}